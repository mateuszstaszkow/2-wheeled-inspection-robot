\documentclass[a4paper,12pt,twoside,openany]{report}
%
% Wzorzec pracy dyplomowej
% J. Starzynski (jstar@iem.pw.edu.pl) na podstawie pracy dyplomowej
% mgr. Błażeja Wincenciaka
% Wersja 3.0 - 10 stycznia 2009
%
\usepackage{polski}
\usepackage[utf8]{inputenc}
\usepackage[pdftex]{graphicx}
\usepackage{tabularx}
\usepackage{array}
\usepackage[polish]{babel}
\usepackage{subfigure}
\usepackage{amsfonts}
\usepackage{verbatim}
\usepackage{indentfirst}
\usepackage[pdftex]{hyperref}


% rozmaite polecenia pomocnicze
% gdzie rysunki?
\newcommand{\ImgPath}{.}

% oznaczenie rzeczy do zrobienia/poprawienia
\newcommand{\TODO}{\textbf{TODO}}


% wyroznienie slow kluczowych
\newcommand{\tech}{\texttt}

% na oprawe (1.0cm - 0.7cm)*2 = 0.6cm
% na oprawe (1.1cm - 0.7cm)*2 = 0.8cm
%  oddsidemargin lewy margines na nieparzystych stronach
% evensidemargin lewy margines na parzystych stronach
\def\oprawa{1.05cm}
\addtolength{\oddsidemargin}{\oprawa}
\addtolength{\evensidemargin}{-\oprawa}

% table span multirows
\usepackage{multirow}
\usepackage{enumitem}	% enumitem.pdf
\setlist{listparindent=\parindent, parsep=\parskip} % potrzebuje enumitem

%%%%%%%%%%%%%%% Dodatkowe Pakiety %%%%%%%%%%%%%%%%%
\usepackage{prmag}   % definiuje komendy opieku,nrindeksu, rodzaj pracy, ...



%%%%%%%%%%%%%%% Strona Tytułowa %%%%%%%%%%%%%%%%%
% To trzeba wypelnic swoimi danymi
\title{TYTUŁ PRACY DYPLOMOWEJ}

% autor(zy)
\author{Zdolny Student}
\nrindeksu{123456}
% wstawienie zdjecia
\zdjecie{\includegraphics[width=4cm]{\ImgPath/zdjecie.png}}

% jeśli wykonawca jest tylko jeden, to usuwamy poniższe polecenia
\authorII{Pracowity Kolega}
\nrindeksuII{654321}
\zdjecieII{\includegraphics[width=4cm]{\ImgPath/zdjecie2.png}}

\opiekun{dr inż. Miły Opiekun}
% opcjonalnie: \konsultant{prof. Dzielny Konsultant}
\terminwykonania{15 lutego 2011} % data złożenia - pokazana na stronie tytułowej
\datawydaniatematu{1 października 2010}
\rokakademicki{2010/2011}

% zakres pracy
\zakres{\begin{enumerate}
 \item Przegląd istniejących rozwiązań
 \item Projekty układu
 \item Budowa robota
 \item Implementacja programów sterujących
 \item Testy i analiza
\end{enumerate}}


% Podziekowanie - opcjonalne
\podziekowania{\noindent
{\Large Podziękowania}
\bigskip

Dziękuję bardzo serdecznie Rodzinie, Wykładowcom, Władzom Politechniki i Promotorowi - dr inż. Krzysztofowi Bieńkowskiemu

\bigskip

{\raggedleft
Mateusz Staszków

}

}

% To sa domyslne wartosci
% - mozna je zmienic, jesli praca jest pisana gdzie indziej niz w ZETiIS
% - mozna je wyrzucic jesli praca jest pisana w ZETiIS
%\miasto{Warszawa}
%\uczelnia{POLITECHNIKA WARSZAWSKA}
%\wydzial{WYDZIAŁ ELEKTRYCZNY}
%\instytut{INSTYTUT ELEKTROTECHNIKI TEORETYCZNEJ\linebreak[1] I~SYSTEMÓW INFORMACYJNO-POMIAROWYCH}
% \zaklad{ZAKŁAD ELEKTROTECHNIKI TEORETYCZNEJ\linebreak[1] I~INFORMATYKI STOSOWANEJ}
%\rodzajpracy{MAGISTERSKA}
%\kierunekstudiow{INFORMATYKA}
%%% koniec od P.W

\opinie{%
  \newpage
\begin{center}
 {\large\bf  Opinia} \\
o pracy dyplomowej magisterskiej wykonanej przez dyplomanta\\
{\bf Zdolnego Studenta i Pracowitego Kolegę} \\
 Wydział Elektryczny, kierunek Informatyka,  Politechnika Warszawska\\
Temat pracy\\
\textit{\bf
TYTUŁ PRACY DYPLOMOWEJ
}\\
\end{center}
\medskip
\noindent
Kierownik pracy dyplomowej: {\bf dr inż. Adam Niewymagający}\\
Ocena pracy dyplomowej: {\bf bardzo dobry}

\medskip

\centerline{\bf Treść opinii}
   Celem pracy dyplomowej panów dolnego Studenta i Pracowitego Kolegi  było
opracowanie systemu pozwalającego symulować  i opartego o oprogramowanie o
otwartych źródłach (ang. Open Source). Jak piszą Dyplomanci, starali się opracować
system, który łatwo będzie dostosować do zmieniających się dynamicznie wymagań,
będzie miał niewielkie wymagania sprzętowe i umożliwiał dalszą łatwą rozbudowę oraz
dostosowanie go do potrzeb.
Przedstawiona do recenzji praca składa się z krótkiego wstępu jasno i
wyczerpująco opisującego oraz uzasadniającego cel pracy, trzech rozdziałów (2-4)
zawierających opis istniejących podobnych
rozwiązań, komponentów rozpatrywanychjako kandydaci do
tworzonego systemu i wreszcie zagadnień wydajności wirtualnych
rozwiązań. Piąty rozdział to opis przygotowanego przez
Dyplomantów środowiska obejmujący opis konfiguracji
środowiska oraz przykładowe ćwiczenia laboratoryjne. Ostatni
rozdział pracy to opis możliwości dalszego
rozwoju projektu. W ramach przygotowania pracy Dyplomanci zebrali i przedstawili w
bardzo przejrzysty sposób duży zasób informacji, co świadczy o dobrej orientacji
w nowoczesnej i ciągle intensywnie rozwijanej tematyce stanowiącej
zakres pracy i o umiejętności przejrzystego przedstawienia tych
wyników. Praca zawiera dwa dodatki, z których pierwszy obejmuje wyniki
eksperymentów i badań nad wydajnością, a drugi to źródła
skryptów budujących środowisko.

 Dyplomanci dość
dobrze zrealizowali postawione przed nimi zadanie,
wykazali się więc umiejętnością zastosowania w praktyce wiedzy
przedstawionej w rozdziałach 2-4.  Uważam, że cele postawione w założeniach pracy zostały pomyślnie
zrealizowane. Proponuję ocenę bardzo dobrą (5).

\vskip 1cm
{
\raggedleft
(data, podpis)\kern1cm

}
  \newpage
  \newpage
\begin{center}
 {\large\bf  Recenzja } \\
pracy dyplomowej magisterskiej wykonanej przez dyplomanta\\
{\bf Zdolnego Studenta i Pracowitego Kolegę} \\
 Wydział Elektryczny, kierunek Informatyka,  Politechnika Warszawska\\
Temat pracy\\
\textit{\bf
TYTUŁ PRACY DYPLOMOWEJ
}\\
\end{center}
\medskip
\noindent
Recenzent: {\bf prof. nzw. dr hab. inż. Jan Surowy}\\
Ocena pracy dyplomowej: {\bf bardzo dobry}
\medskip


\centerline{\bf Treść recenzji}
   Celem pracy dyplomowej panów dolnego Studenta i Pracowitego Kolegi  było
opracowanie systemu pozwalającego symulować  i opartego o oprogramowanie o
otwartych źródłach (ang. Open Source). Jak piszą Dyplomanci, starali się opracować
system, który łatwo będzie dostosować do zmieniających się dynamicznie wymagań,
będzie miał niewielkie wymagania sprzętowe i umożliwiał dalszą łatwą rozbudowę oraz
dostosowanie go do potrzeb.
Przedstawiona do recenzji praca składa się z krótkiego wstępu jasno i
wyczerpująco opisującego oraz uzasadniającego cel pracy, trzech rozdziałów (2-4)
zawierających bardzo solidny i przejrzysty opis: istniejących podobnych
rozwiązań (rozdz. 2), komponentów rozpatrywanychjako kandydaci do
tworzonego systemu (rozdz. 3) i wreszcie zagadnień wydajności wirtualnych
rozwiązań, zwłaszcza w kontekście współpracy  kilku elementów
 sieci (rozdział 4). Piąty rozdział to opis przygotowanego przez
Dyplomantów środowiska obejmujący opis konfiguracji
środowiska oraz przykładowe ćwiczenia laboratoryjne (5 ćwiczeń). Ostatni, szósty
rozdział pracy to krótkie zakończenie, które wylicza także możliwości dalszego
rozwoju projektu. W ramach przygotowania pracy Dyplomanci zebrali i przedstawili w
bardzo przejrzysty sposób duży zasób informacji o narzędziach, Rozdziały 2, 3 i 4 świadczą o dobrej orientacji
w nowoczesnej i ciągle intensywnie rozwijanej tematyce stanowiącej
zakres pracy i o umiejętności syntetycznego, przejrzystego przedstawienia tych
wyników. Drobne  mankamenty tej części pracy to zbyt skrótowe omawianie
niektórych zagadnień technicznych, zakładające dużą początkową wiedzę czytelnika
i dość niestaranne podejście do powołań na źródła.
Utrudnia to w pewnym stopniu czytanie pracy i zmniejsza jej wartość dydaktyczną
(a ta zdaje się być jednym z celów Autorów), ale jest zrekompensowane zawartością
merytoryczną. Praca zawiera dwa dodatki, z których pierwszy obejmuje wyniki
eksperymentów i badań nad wydajnością, a drugi to źródła
skryptów budujących środowisko. Praca
zawiera niestety dość dużą liczbę drobnych błędów redakcyjnych, ale nie wpływają
one w sposób istotny na na jej czytelność i wartość. W całej pracy przewijają
się samodzielne, zdecydowane wnioski Autorów, które są wynikiem własnych i
oryginalnych badań.  Rozdział 5 i dodatki pracy przekonują mnie, że Dyplomanci dość
dobrze zrealizowali postawione przed nimi zadanie. Pozwala to stwierdzić, że
wykazali się więc także umiejętnością zastosowania w praktyce wiedzy
przedstawionej w rozdziałach 2-4. Kończący pracę rozdział szósty świadczy o
dużym (ale moim zdaniem uzasadnionym) poczuciu własnej wartości i jest
świadectwem własnego, oryginalnego spojrzenia na tematykę przedstawioną w pracy
dyplomowej. Uważam, że cele postawione w założeniach pracy zostały pomyślnie
zrealizowane. Proponuję ocenę bardzo dobrą (5).

\vskip 1cm
{
\raggedleft
(data, podpis)\kern1cm

}
}


\begin{document}
\maketitle

%-----------------
% Wstęp
%-----------------
\chapter{Wstęp}
W ciągu ostatnich lat możemy zaobserwować dynamiczny rozwój sieci komputerowych. W ogólności można powiedzieć, że ilość przesyłanych w nich danych dubluje się w ciągu roku. Jednocześnie możemy zaobserwować migrację transmisji głosowych (sieci telefoniczne) i transmisji multimedialnych (sieci telewizji kablowej) do sieci komputerowych, co jest zgodne z teorią konwergencji. Znaczenie sieci rośnie również w przypadku transakcji bankowych i finansowych, a także w stosunku do sieciowych systemów komputerowych (podnoszenie stacji roboczych po sieci, praca zdalna).

W każdym zespole opracowującym nową technologię komunikacji sieciowej, głównym zadaniem jest zapewnienie bezpieczeństwa przesyłanych informacji. Dodatkowym wymaganiem jest stworzenie mechanizmów, dzięki którym możliwa będzie kontrola przesyłanych informacji, struktury użytego protokołu oraz uniemożliwienie lub utrudnienie wykorzystania kanału transmisji do przesyłania niekotrolowanych informacji steganograficznych. 

Zastosowanie metod steganograficznych wprowadza dodatkowy, tajny kanał informacji w ramach legalnego strumienia danych. Rozwój technik steganograficznych nakłada coraz to większe wymagania na algorytmy dbające o bezpieczeństwo informacji. Naruszeń bezpieczeństwa nie wystarczy szukać jedynie w samej treści informacji. We współczesnych systemach należy także poddać analizie miejsca nie przeznaczone do przesyłania danych, które zapewniają jedynie poprawność działania protokołu.

Technika steganograficzna jest szczególnie atrakcyjna dla osób chcących ukryć swoje działania, takich jak twórcy ,,złośliwego'' oprogramowania. Pozwala ona kontrolować i wymieniać informacje z systemami, które zostały przejęte przez włamywacza, bez obawy wykrycia faktu istnienia takiej komunikacji. Metody te są także stosowane przez osoby mające wgląd do wartościowych informacji, których nie mogą w bezpośredni sposób przekazać poza chronione systemy. Ukrycie faktu tajnej komunikacji jest więc szczególnie atrakcyjne dla szpiegów politycznych, wojskowych oraz gospodarczych.

Z wyżej wymienionych powodów, w instytucjach zawierających dane ściśle tajne, w ośrodkach wojskowych lub firmach, które przeznaczyły znaczne środki finansowe na opracowanie nowych technologii, użycie steganografii rozważane jest jako jedno z czołowych zagrożeń dla bezpieczeństwa informacji. Z tego względu tak ważne są metody stegoanalizy, które potrafią zdemaskować fakt użycia steganografii i nie dopuścić do wycieku ważnych informacji. 

  %-----------------
  % Postawienie zadania
  %-----------------
\section{Postawienie zadania}

Zadaniem pracy jest przedstawienie algorytmów steganograficznych, z naciskiem położonym na te, które wykorzystują jako kanał transmisji protokół TCP. Częścią praktyczną pracy jest opracowanie oraz zaimplementowanie efektywnego kanału steganograficznego opartego o protokół TCP. Dodatkowo zaprezentowane zostaną metody wykrywania steganografii w odniesieniu do istniejących rozwiązań oraz  zaproponowanego algorytmu steganograficznego.

Cele jakie należy osiągnąć, aby skonstruować efektywny kanał steganograficzny oparty na protokole TCP/IP to:
\begin{enumerate}
	\item modyfikacja pól nagłówków TCP/IP w taki sposób, aby nie wpłynąć na mechanizmy zawarte w protokole TCP/IP, szczególnie na poprawne zachowanie się sesji TCP,
	\item kanał steganograficzny nie powinien być zawarty w polach nagłówka, które zgodnie ze specyfikacją lub powszechnymi praktykami zawierają ściśle określone wartości, czyli są np. zerowe,
	\item modyfikacja pól powinna być trudna do wykrycia, tak aby techniki służące do demaskowania ukrytego kanału wymagały znacznego nakładu pracy,
	\item stosowanie steganografii w nagłówkach TCP/IP nie powinno narzucać dodatkowych wymagań czasowych na generację pakietów, co byłoby łatwo zauważalne przez osoby nadzorujące ruch w sieci,
	\item w sytuacji, gdy kanał steganograficzny będzie wykorzystywany do transmisji dużych ilości informacji, należy wziąć pod uwagę dodatkowe narzuty obliczeniowe związane z wyznaczaniem pól nagłówków TCP/IP,
	\item dodatkowym wymaganiem narzuconym na kanał steganograficzny jest zachowanie poufności danych, nawet w przypadku wykrycia tajnej informacji, co sprowadza się do wykorzystania kryptografii,
	\item przesyłanie tajnych danych powinno odbywać się w ramach istniejącego, legalnego strumienia danych, bez konieczności generowania własnego ruchu TCP/IP (cecha opcjonalna).
\end{enumerate}

%-----------------
% Steganografia
%-----------------
\chapter{Steganografia}
Wywodzące się z greki słowo ,,steganografia'' oznacza ,,ukryte pismo'' (\textit{steganos} - ukryty, tajny; \textit{graphein} - pisać, malować), co w odniesieniu do kanału informacyjnego oznacza przesyłanie danych w taki sposób, aby osoby postronne mające wgląd do danych nie mogły stwierdzić istnienia w nich ukrytej informacji. Cały mechanizm steganografii opiera się na zasadzie ukrycia informacji w tych częściach wiadomości, które nie służą do przekazywania informacji lub których modyfikacja nie wpływa na treść głównego przekazu.

W celu przesłania informacji za pomocą steganografii należy utworzyć kanał steganograficzny, zdefiniowany \cite{USDoD}  jako: ,,każdy kanał komunikacyjny, który może być wykorzystany przez stronę do przesłania informacji w sposób naruszający politykę bezpieczeństwa systemu''. Metoda ta wykorzystuje fakt przesłania danych w sposób i w miejscach, które zgodnie z protokołem do tego nie służą, narażając system na nieautoryzowany przesył informacji.

Steganografia w znaczącym stopniu różni się od kryptografii, która nie dba o zatajenie istnienia przekazu, a jedynie o jego integralność oraz uniemożliwienie stronom trzecim poznanie treści przekazu. Oczywiście najlepszą techniką jest połączenie steganografii z kryptografią. Takie podejście pozwala zabezpieczyć się przed sytuacją, w której strona nadzorująca transmisję, nawet w przypadku odkrycia przekazu steganograficznego nie może go odczytać ze względu na siłę zastosowanej kryptografii.
  %-----------------
  % Historia
  %-----------------
\section{Historia}
Pomimo, że pierwsze wzmianki o steganografii, a dokładnie o ukrytych kanałach w odniesieniu do systemów informatycznych notuje się na lata siedemdziesiąte XX wieku \cite{FirstCC}, to przykłady użycia steganografii sięgają starożytności. W literaturze powtarzają się opisy przekazywania tajnej informacji poprzez wytatuowanie jej na ogolonej głowie posłańca, który po odrośnięciu włosów był wysyłany z mało znaczącą wiadomością do armii swojego dowódcy. Każdy kto natknął się na posłańca miał wgląd do nieważnej wiadomości, niepodejrzewając nawet istnienia sekretnej informacji w postaci tatuażu.

Przykłady z historii odnoszą się także do bardziej współczesnych czasów. Wiele z metod steganografii było stosowanych podczas II Wojny Światowej (np. mikro-kropki) a także w latach Zimnej Wojny. Wiadomo także, że wielu agentów służb wywiadowczych, a szczególnie podwójnych agentów, przekazywało obcym państwom informację wykorzystując steganografię. Przykładem może tu być sprawa szpiega FBI Roberta Hanssena \cite{Hanssen}, który przy pomocy technik steganograficzny przez około dekadę przekazywał tajne informacje służbom KGB.

Rozdziały \ref{sectionSteganografiaWObiektachMultimedialnych} oraz \ref{chapterSteganografiaWRuchuTCPIP} opisują nowoczesne podejście do steganografii wykorzystujące współczesne kanały informacyjne. 
  %-----------------
  % Pojęcia
  %-----------------
\section{Pojęcia}
W celu zdefiniowania kanału steganograficznego oraz opisania transmisji z wykorzystaniem takiego kanału należy omówić jego części składowe:
\begin{itemize}
	\item dane do ukrycia, tajne dane - informacja jaką należy przesłać między uczestnikami komunikacji, tak aby strony trzecie nie miały do niej wglądu,
	\item dane nośne, wiadomość zakrywająca - wiadomość, w której ukryte zostaną tajne dane; przesyłanie wiadomości zakrywających musi być dozwolone w danym kanale informacyjnym i nie powinno wzbudzać podejrzeń,
	\item funkcja steganograficzna - funkcja przekształcająca dane do ukrycia oraz wiadomość zakrywającą w jedną połączoną wiadomość,
	\item dane z ukrytą wiadomością - dane zawierające ukrytą informację a jednocześnie wykazujące cechy danych nośnych,
	\item nadzorca komunikacji, wartownik - mechanizm mający pełen wgląd do wiadomości przekazywanej między stronami komunikacji, świadomy struktury komunikatów i potrafiący wykrywać występujące w nich anomalie,
	\item kanał komunikacyjny - kanał zestawiony pomiędzy nadawcą a odbiorcą, zapewniający przepływ informacji, do którego wgląd ma nadzorca komunikacji,
	\item odwrotna funkcja steganograficzna - funkcja przekształcająca dane z ukrytą wiadomością na tajne dane,
	\item klucz kryptograficzny - klucz znany tylko obu stronom komunikacji, służący do zabezpieczenia tajnej informacji metodami kryptografii symetrycznej przed ewentualnością złamania funkcji steganograficznej.
\end{itemize}
  %-----------------
  % Schemat komunikacji steganograficznej
  %-----------------
\section{Schemat komunikacji steganograficznej}
\label{sectionSchematKomunikacjiSteganograficznej}
Podstawowy scenariusz, powszechny w literaturze na temat steganografii, odnosi się do sytuacji opisanej w \cite{PrisonersProblem}. Dwóch więźniów (w naszym przypadku Alicja(\tech{A}) i Bob(\tech{B})) zamknięci są w dwóch odrębnych celach. Mogą się ze sobą kontaktować, jednak ich cała korespondencja przechodzi przez ręce Wartownika (\tech{W}). Ma on pełen wgląd do przekazywanych informacji, więc może przechwycić wszelkie przekazywane tajemnice, a dodatkowo w razie podejrzeń może nie dopuścić do komunikacji\footnote{podejrzana informacja jest tu analogią do stosowania kryptografii przez więźniów}. W takim przypadku w celu przekazania ważnych informacji \tech{A} i \tech{B} muszą posłużyć się pewnego rodzaju podstępem. Muszą tak sformułować treść przekazu, aby \tech{W} nie rozróżnił ,,niegroźnej'' wiadomości od wiadomości z ukrytym przekazem. Dlatego też przekazują wiadomość, w której prawdziwa treść możliwa jest do odczytania po złożeniu kolejno każdej np.  drugiej litery z każdego wyrazu.
\begin{figure}[!htbp]
	\begin{center}
\centering
\includegraphics[scale=0.4]{\ImgPath/rys/schemat_komunikacji.png}
\end{center}
	\caption{Schemat komunikacji steganograficznej}
	\label{schematKomunikacji}
\end{figure}

Przedstawioną tak sytuację pokazuje rysunek \ref{schematKomunikacji}\footnote{sporządzony na podstawie \cite{schematKomunikacjiPrzypis}, rysunek 1, strona 3}. \tech{A} próbuje przesłać tajną informację \tech{E} do \tech{B}. Cała komunikacja odbywa się przez kanał publiczny, kontrolowany przez \tech{W}. W celu ukrycia faktu komunikacji \tech{A} stara się ukryć tajny przekaz w informacji \tech{C}. W celu uzyskania skutecznej steganografii \tech{W} nie może rozróżnić informacji poprawnej, nie zawierającej tajnych danych, od informacji \tech{S}, która zawiera tajną informację. W celu dodatkowego zabezpieczenia przekazu, \tech{A} i \tech{B} mogą korzystać z funkcji kryptograficznej zabezpieczającej przekazywane informacje. Można tu wykorzystać metody kryptografii symetrycznej (ustalony klucz kryptograficzny \tech{K}) lub niesymetrycznej (klucz publiczny \tech{K}$_{pub}$ i klucz prywatny \tech{K}$_{pryw}$).

Stosowanie technik kryptograficznych wpływa na poprawę bezpieczeństwa przesyłanej informacji, jednak należy pamiętać o nieporządnych cechach jakie mogą one wywołać. W większości przypadków umieszczenie tajnej informacji steganograficznej w przekazie wiąże się z zamianą istniejącej już nieważnej części informacji. Jednak każda porcja usuniętej informacji może mieć pewną charakterystyczną postać lub specyficzny histogram. Zastosowanie funkcji kryptograficznej w stosunku do tajnej informacji zmienia ją, a wynikowy rozkład bitów jest nieprzewidywalny i w większości przypadków różny od standardowych histogramów określonych dla podmienianych części wiadomości.
  %-----------------
  % Stegoanaliza
  %-----------------
\section{Stegoanaliza}
Stegoanaliza to nauka zajmująca się wykrywaniem istnienia ukrytych informacji w kanałach komunikacyjnych. Nie zawsze prowadzi to do odkrycia dokładnej treści ukrytego przekazu, a w większości przypadków polega jedynie na wskazaniu istnienia ukrytego kanału steganograficznego.

Możliwość wykrycia kanału steganograficznego sprowadza się do analizy różnych części wiadomości lub strumienia danych w celu wykrycia anomalii. Takie podejście wynika z faktu, że tajna informacja ukryta jest w miejscach nie przeznaczonych do przesyłania informacji lub na miejscu danych, które są w pewien sposób nadmiarowe (np. dla zmysłów człowieka). Można wskazać dwa podstawowe sposoby wykrywania anomalii:
\begin{itemize}
	\item pierwsze podejście opiera się na przebadaniu wszystkich części informacji (np. pól nagłówka TCP/IP), których struktura jest w pełni przewidywalna lub których wartości są zdefiniowane przez standardy lub powszechne praktyki; ważne jest także sprawdzenie czy występują wartości nadmiarowe oraz czy elementy sygnalizujące wystąpienie dodatkowych danych mają faktyczne pokrycie w danych,
	\item drugą metodą jest porównanie wartości części wiadomości (np. pól nagłówka TCP/IP) i zaklasyfikowanie ich jako prawdopodobnych lub nie dla danego systemu bądź protokołu; takie podejście może być stosowane do wartości ściśle określonych, takich jak wymienione w pierwszym punkcie, jednak można je także stosować do wartości które są pseudolosowe lub których histogram jest charakterystyczny; w celu realizacji tej metody warto posłużyć się sieciami neuronowymi takimi jak SVM i RSVM, zdolnymi rozpoznawać wzorce i separować dane.
\end{itemize}
  %-----------------
  % Metody tworzenia steganografii
  %-----------------
\section{Metody tworzenia steganografii oraz rodzaje ukrytych kanałów}
Przesłanie danych za pomocą przekazu steganograficznego wiąże się w większości przypadków z umieszczeniem dodatkowej informacji w wiadomości. Odbywa się to za pomocą podmiany tej części wiadomości (nagłówka TCP/IP), która wykazuje cechy nadmiarowości lub której (kontrolowana) zmiana nie prowadzi do przerwania transmisji. Pewną podgrupą może być w tym przypadku wykorzystanie pól oryginalnie pustych (zerowych) lub niewykorzystywanych w istniejących implementacjach.

Kanały steganograficzne można podzielić na dwa zasadniczne typy\cite{SweetyPresentation}:
\begin{itemize}
	\item kanał pojemnościowy (ang. storage channel) - informacja zawarta w częściach wiadomości, polach nagłówka,
	\item kanał czasowy (ang. timing channel) - informacja zawarta w czasach wystąpienia danych zdarzeń, np. przesłania pakietu TCP/IP.
\end{itemize}
W przypadku sieci pakietowych można także połączyć dwa typu kanałów steganograficznych, tworząc kanał mieszany, w którym jeden z typów (np. pojemnościowy) będzie wykorzystywany do przekazywania informacji, a drugi (np. czasowy) do sygnalizacji tego zdarzenia.

Większość opracowanych programów służących do przesyłania danych z wykorzystaniem steganografii opiera się na kanałach pojemnościowych. Wynika to z faktu, że kanały czasowe narzucają pewne ograniczenia na generację pakietów TCP/IP przez co ich wykrycie staje się prostsze.

Dodatkowo należy zauważyć, że w sieciach pakietowych można skonstruować abstrakcyjny kanał steganograficzny, w którym do przesyłania tajnych danych lub/i obsługi protokołu steganograficznego wykorzystywane są różne pola nagłówka. Zmiana wykorzystania danego pola może być dynamiczna, zależna od wymaganej przepustowości lub w celu zminimalizowania wykrycia kanału steganograficznego. 
  %-----------------
  % Cechy kanału steganograficznego
  %-----------------
\section{Cechy kanału steganograficznego}
Każdy kanał steganograficzny posiada trzy cechy, które decydują o jego przydatności w danej sytuacji:
\begin{enumerate}
	\item pojemność (przepustowość) - określa jaką porcję informacji możemy przesłać w danej wiadomości nośnej; w przypadku steganografii w TCP/IP, wyrażana jest w bitach na sekundę, bitach na pakiet lub bitach na sesję TCP; przepustowość odgrywa ważną rolę w przypadku konieczności przekazania dużej ilości informacji, jednak należy pamiętać, że to przeważnie prowadzi do ułatwionej detekcji steganografii,
	\item bezpieczeństwo - określa jak łatwo jest uzyskać dostęp do przekazywanej tajnej informacji w przypadku poznania mechanizmu tworzenia przekazu steganograficznego; dodatkowym mechanizmem zwiększającym bezpieczeństwo może być używanie znanych tylko sobie zmiennych pseudolosowych lub modyfikacji algorytmu\footnote{jest to znane jako ,,bezpieczeństwo przez zatajenie'' (ang. security through obscurity) i powinno być używane tylko jako dodatkowy element systemu zabezpieczeń},
	\item krzepkość (ang. robustness) - określa stopień w jakim możemy zmodyfikować przekaz nie uszkadzając zawartej w nim informacji steganograficznej; niestety w przypadku steganografii naruszenie kanału (pola) zawierającego przekaz steganograficzny przeważnie wiąże się z utratą tajnego przekazu.
\end{enumerate}

  %-----------------
  % Steganografia w obiektach multimedialnych
  %-----------------
\section{Steganografia w obiektach multimedialnych}
\label{sectionSteganografiaWObiektachMultimedialnych}
Pomimo, że steganografia ma zastosowanie prawie w każdej formie komunikacji, w latach 90-tych zyskała ona powodzenie jako technika ukrywania informacji w obiektach multimedialnych. Wynika to przede wszystkim z powszechności tego rodzaju przekazu, jego rozmiarów oraz prostoty obsługi programów do ukrywania informacji w obiektach multimedialnych, takich jak obraz, dźwięk i wideo. Dodatkowym atutem przy zastosowaniu tych metod jest stosunek ukrytej informacji do oryginalnego przekazu, sięgający w ekstremalnych sytuacjach 50\%, bez zauważalnego pogorszenia się jakości przekazywanych danych.

Użycie steganografii w treściach multimedialnych sprowadza się do takiego manipulowania danymi, aby plik wynikowy zawierał dodatkowe informacje, a jednocześnie nie był rozróżniany przez zmysły człowieka w porównaniu z oryginałem.

Jedną z najszerzej omawianych form steganografii w obiektach multimedialnych jest ukrywanie informacji w plikach graficznych. Istnieje wiele rozwiązań, zarówno bezpłatnych, o otwartym kodzie jak i komercyjnych. Przykładami mogą tu być takie programy jak Outguess, JPHide, StegHide. Istnieją różne techniki ukrywania informacji w plikach graficznych. Najprostszym rozwiązaniem jest podmiana najmniej znaczących bitów opisujących kolor danego piksela. Możliwe jest też zastosowanie dyskretnej transformaty kosinusowej.

W przypadku wybrania jako wiadomości nośnej pliku audio, możemy także zastosować metodę podmiany najmniej znaczących bitów. Dodatkowo stosowane są metody ukrywania tajnych wiadomości poprzez rozszerzanie spektrum danego nagrania, czy też dodawanie echa. Przykładem narzędzia do tworzenia wiadomości steganograficznych może być UnderMP3Cover, MP3Stego czy S-Tools\footnote{\url{http://www.stegoarchive.com}}.

Kolejnym przykładem wykorzystania jako pliku nośnego obiektu multimedialnego jest plik wideo. Dodatkowa informacja może być przekazana przy użyciu dyskretnej transformaty kosinusowej. Jako przykładowe implementacje można podać StegoVideo.

Istnieje kilka technik umożliwiających wykrycie lub usunięcie steganografii zastosowanej w obiektach multimedialnych. Pierwszym podejściem, choć przeważnie trudnym do zastosowania, jest użycie oryginalnego pliku jako wzorca do porównania z przechwyconą wersją. W przypadku plików graficznych lub wideo możliwe jest użycie analizatorów statystycznych, które mogą wykryć anomalie występujące w histogramach tych wiadomości.

Zamiast wykrywać istnienie steganografii, częstym podejściem jest jej ograniczanie lub ,,ślepe'' usuwanie z wiadomości tych danych, które mogą być nośnikiem kanału steganograficznego. W przypadku plików multimedialnych najlepszym sposobem uzyskania takiego efektu jest przekodowanie pliku na inny standard i powrót do standardu wejściowego. Przeważnie zmiany w jakości plików są niezauważalne, a użycie konwersji sprowadza się do takiej zmiany bitów, która niszczy zawartą w nich steganografię.

W przypadku plików multimedialnych użycie steganografii jest pomocne w ochronie praw autorskich, przez stosowanie jej jako cyfrowych znaków wodnych. Niestety, tak jak zostało to wcześniej przedstawione w trakcie konwersji wiele z zakodowanej informacji ginie bezpowrotnie. Skutkiem tego może być pogorszenie jakości pliku multimedialnego, ale także usunięcie z niego cyfrowego znaku wodnego.

Itd., itd., itd. ...

\chapter{Steganografia w ruchu TCPIP}
\label{chapterSteganografiaWRuchuTCPIP}

Itd., itd., itd ...

%-----------------
% Wnioski 
%-----------------
\chapter{Wnioski}

Protokół TCP/IP jest najbardziej rozpowszechnionym i używanym protokołem komunikacji między systemami w sieci Internet oraz w sieciach intranet. Niestety został on opracowany na początku lat siedemdziesiątych, gdy problemy bezpieczeństwa informacji nie stały na pierwszym miejscu. Ciągły wzrost działań przestępczych w sieci Internet, w tym wymiana nielegalnych treści, prowadzi do stosowania coraz to nowszych technik zabezpieczających. Z tego względu obserwuje się działania mające na celu wprowadzenie tajnej komunikacji między przejętymi systemami, tak aby nie wzbudzić ostrzeżeń w analizatorach sieciowych. Taka ukryta komunikacja odbywa się z wykorzystaniem steganografii.

Wprowadzenie steganografii do niskich warstwach stosu TCP/IP umożliwia obejście wielu filtrów nałożonych na warstwy wyższe. Większość sieci oparta jest na protokołach rodziny TCP/IP, przez co nie można zabronić ich używania. Możliwa jest jedynie kontrola poprawności semantyki protokołów TCP/IP, a także ewentualna ingerencja w przekazywane wartości, z uwzględnieniem stanowości niektórych pól.

Opracowany schemat generacji początkowych numerów sekwencyjnych w jak najlepszy sposób odzwierciedla oryginalny proces zachodzący w stosie sieciowym systemu Linux. W większości przypadków występujących w rzeczywistych sieciach i systemach, numery wygenerowane przy pomocy \tech{Shushi} nie byłyby rozróżnialne od numerów wygenerowanych przez stos sieciowy systemu.

Jeżeli proces generacji wartości użytych do przekazania danych steganograficznych zostanie oparty o oryginalne mechanizmy używane do ich generacji, to pasywny analizator sieciowy nie będzie w stanie wykryć istnienia anomalii. Różnice możliwe są do zaobserwowania w przypadku zaistnienia specyficznych sytuacji występujących dla danej implementacji protokołu. W przypadku zastosowania pasywnego analizatora wymaga to jednak oczekiwania na taką sytuację. Z przeprowadzonych testów wynika, że lepszym podejściem jest zastosowanie analizatorów aktywnych, które posiadają wiedzę na temat testowanych systemów oraz ich chrakterystycznych cech implementacji. Skonstruowanie takiego analizatora jest zadaniem stosunkowo prostym a daje bardzo wysoką skuteczność.

Z przeprowadzonych testów wynika, że celowe jest prowadzenie dalszych prac w następujących obszarach:
\begin{itemize}
 \item dokładniejszy mechanizm generacji wartości mikrosekund
 \item wprowadzenie algorytmów zdolnych wykryć i uniemożliwić działanie analizatora aktywnego
\end {itemize}

Jeżeli powyższe punkty nie zostaną spełnione, analizatory aktywne będą w stanie wykryć istnienie modułu steganograficznego opartego na początkowych numerach sekwencyjnych.

Rozwój opracowanego rozwiązania steganograficznego jest możliwy poprzez wprowadzenie elementów takich jak:
\begin{itemize}
 \item obsługa innych, przyszłościowych protokołów sieciowych, takich jak SCTP (ang. Stream Control Transmission Protocol)\cite{RFC2960}
 \item zapewnienie dwustronnej komunikacji z wykorzystaniem numerów potwierdzenia \tech{ACK}
 \item przeniesienie implementacji do innych systemów operacyjnych
\end{itemize}

Wraz ze wzrostem przepustowości urządzeń sieciowych (obecnie 10Gb/s i więcej) wzrasta problem analizy przepływających danych w czasie rzeczywistym. Analizatory sieciowe muszą w coraz krótszym czasie zbadać coraz większy strumień danych (miliony pakietów na sekundę). Jednak problem wzrostu prędkości sieci utrudnia zadanie także osobom implementującym kanały steganograficzne w protokole TCP/IP. Coraz więcej operacji wyższych warstw stosu sieciowego przenoszonych jest do układów scalonych interfejsów sieciowych. Taka technologia znana jest pod skrótem TOE (ang. TCP Offload Engine) i odnosi się przede wszystkim do sprzętowej generacji sum kontrolnych oraz mechanizmu TSO (ang. TCP segmentation offload). W następnych latach spodziewane jest przenoszenie kolejnych elementów stosu sieciowego TCP/IP do implementacji sprzętowych.

Ze względu na rozwój systemów zabezpieczających ruch sieciowy oraz wzrost bezpieczeństwa systemów operacyjnych, w kolejnych latach wzrośnie także wykorzystanie technik steganograficznych przez grupy przestępcze działające w ramach Internetu. Z tego powodu poznanie technik steganograficznych oraz wypracowanie metod obrony i wykrywania takiej komunikacji jest bardzo ważne.

%-----------------
% Dodatki 
%-----------------
\appendix
\chapter{Porównanie numerów ISN jądra Linux i modułu \tech{Shushi}}
\begin{figure}[!htbp]
	\begin{center}
\centering
\includegraphics[scale=0.21]{\ImgPath/rys/IPPortConstData.pdf}
\end{center}
	\caption{Numery ISN wygenerowane przez jądro oraz \tech{Shushi}, stałe numery IP oraz porty TCP, stałe dane dla \tech{Shushi}, serie po około 2800 próbek.}
	\label{IPPortConstData}
\end{figure}

\begin{figure}[!htbp]
	\begin{center}
\centering
\includegraphics[scale=0.21]{\ImgPath/rys/IPPortConstDataDiff.pdf}
\end{center}
	\caption{Różnice pomiędzy kolejnymi numerami ISN wygenerowanymi przez jądro oraz \tech{Shushi}, stałe numery IP oraz porty TCP, stałe dane dla \tech{Shushi}, serie po około 60000 próbek.}
	\label{IPPortConstDataDiff}
\end{figure}

\begin{figure}[!htbp]
	\begin{center}
\centering
\includegraphics[scale=0.21]{\ImgPath/rys/IPPortRandData.pdf}
\end{center}
	\caption{Numery ISN wygenerowane przez jądro oraz \tech{Shushi}, stałe numery IP oraz porty TCP, losowe dane dla \tech{Shushi}, serie po około 860 próbek.}
	\label{IPPortRandData}
\end{figure}

\begin{figure}[!htbp]
	\begin{center}
\centering
\includegraphics[scale=0.21]{\ImgPath/rys/IPPortRandDataDiff.pdf}
\end{center}
	\caption{Różnice pomiędzy kolejnymi numerami ISN wygenerowanymi przez jądro oraz \tech{Shushi}, stałe numery IP oraz porty TCP, losowe dane dla \tech{Shushi}, serie po około 60000 próbek.}
	\label{IPPortRandDataDiff}
\end{figure}

\begin{thebibliography}{99}
\addcontentsline{toc}{chapter}{Bibliografia}
\bibitem{Stevens}{W. R. Stevens, G. R. Wright, ,,Biblia TCP/IP tom 1'', RM, 1998.}
\bibitem{USDoD}{U. S. Department Of Defense, ,,Trusted Computer System Evaluation Criteria'', 1985.}
\bibitem{FirstCC}{B. W. Lampson, ,,A note on the confinment problem'', w ,,Proc. of the Communications of the ACM'', październik 1973, numer 16:10,\newline strony 613-615.} 
\bibitem{PrisonersProblem}{G. J. Simmons, ,,The prisoners' problem and the subliminal channel'', w ,,Advances in Cryptology: Proceedings of Crypto 83 (D. Chaum, ed.)'', strony 51-67, Plenum Press, 1984.}
\bibitem{Kerckhoff}{ A. Kerckhoffs, ,,La Cryptographie Militaire (Military Cryptography)'', J. Sciences Militaires, luty 1883.} 
\bibitem{Hanssen}{A. Havill, ,,The Spy Who Stayed Out In The Cold: The Secret Life of Double Agent Robert Hanssen'', St. Martin's Press, 2001.}
\bibitem{schematKomunikacjiPrzypis}{C.Cachin, ,,An Information-Theoretic Model for Steganography'', w ,,Information and Computation'', 4 marzec 2004.}
\bibitem{SweetyPresentation}{S.Chauhan, ,,Embedding Covert Channels into TCP/IP'', 7th Information Hiding Workshop, czerwiec 2005.}
\bibitem{RFC793}{Information Sciences Institute, University of Southern California, ,,Transmission Control Protocol'', RFC793, wrzesień 1981.}
\bibitem{RFC1323}{V. Jacobson, R. Braden, D. Borman, ,,TCP extensions for high performance'', RFC1323, maj 1992.}
\bibitem{RFC1948}{S. Bellovin, ,,Defending against sequence number attacks.'', RFC1948, IETF, 1996.}
\bibitem{RFC2960}{R. Stewart, Q. Xie, K. Morneault, C. Sharp, H. Schwarzbauer, T. Taylor, I. Rytina, M. Kalla, L. Zhang, V. Paxson, „Stream Control Transmission Protocol”, RFC2960, Network Working Group, październik 2000.}
\bibitem{Rowland}{C. H. Rowland, ,,Covert Channels in the TCP/IP Protocol Suite'', First Monday, 1997. \newline \url{http://www.firstmonday.dk/issues/issue2\_5/rowland/}}
\bibitem{LOKI}{Alhambra, daemon9, ,,Project Loki: ICMP Tunneling'', Phrack Magazine, Issue 49. \url{http://phrack.org}}
\bibitem{LOKI2}{daemon9, ,,LOKI2'', Phrack Magazine, Issue 51. \url{http://phrack.org}}
\bibitem{RWWWS}{van Hauser, Reverse WWW Shell,  THC, The Hacker's Choice.\newline \url{www.thc.org}}
\bibitem{CCdetectionSVM}{T. Sohn, J. Seo, J. Moon, ,,A Study on the Covert Channel Detection of TCP/IP Header Using Support Vector Machine'', Volume 2836 of Lecture Notes in Computer Science., Springer-Verlag (2003) 313-324.}
\bibitem{LOKIdetectionSVM}{T. Sohn, T. Noh, J. Moon, ,,Support Vector Machine Based ICMP Covert Channel Attack Detection'', Volume 2836 of Lecture Notes in Computer Science., Springer-Verlag, 2003, strony 461-464.}
\bibitem{devcc}{J. Giffin, R. Greenstadt, P. Litwack, R. Tibbetts, ,,Covert messaging in TCP'', w Dingledine, Privacy Enhancing Technologies. Volume 2482 of Lecture Notes in Computer Science., Springer-Verlag (2002) 194-208. \url{http://www.mit.edu/\textasciitilde gif/covert-channel/}}
\bibitem{ActiveWardens}{G. Fisk, M. Fisk, Ch. Papadopoulos, J. Neil, ,,Eliminating Steganography in Internet Traffic with Active Wardens'', 5th International Workshop on Information Hiding, październik 2002.} 
\bibitem{JR}{J. Rutkowska, ,,The Implementation of Passive Covert Channels in Linux Kernel'', Chaos Communication Congress, grudzień 2004.}
\bibitem{LinuxNetwork}{Ch. Benvenuti, ,,Understanding Linux Network Internals'', O'Reilly,\newline grudzień 2005.}
\bibitem{p55}{kossak, ,,Building Into The Linux Network Layer'', Phrack Magazine, Issue 55. \url{http://phrack.org}}
\bibitem{ML}{Steven J.Murdoch and Stephen Lewis, ,,Embedding Covert Channels into TCP/IP'', University of Cambridge, Computer Laboratory,\newline 29 lipec 2005.}
\bibitem{NvsNN}{Eugene Tumoian, Maxim Anikeev, ,,Detecting NUSHU Covert Channels Using Neural Networks'', Taganrog State University of Radio Engineering, Department of Information Security.}
\bibitem{p58}{mayhem, ,,IA32 Advanced Function Hooking'', Phrack Magazine,\newline Issue 58. \url{http://phrack.org}}
\bibitem{p61}{bioforge, ,,Hacking the Linux Kernel Network Stack'', Phrack Magazine, Issue 61. \url{http://phrack.org}}
\bibitem{kernelMEM}{Robert Love, ,,Kernel Korner - Allocating Memory in the Kernel'',\newline 1 grudzień 2003.}

\end{thebibliography}

\zakonczenie  % wklejenie recenzji i opinii

\end{document}
%+++ END +++
